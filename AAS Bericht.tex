\documentclass[10pt]{article}
\usepackage[german]{babel}

%\usepackage[latin9]{inputenc} 



\usepackage{graphicx}
\usepackage{amssymb}
\usepackage{mathtools}
\usepackage{gensymb}
\usepackage{a4}
\usepackage[left=3cm,right=3cm,top=1cm,bottom=2cm, includehead]{geometry}
\usepackage[applemac]{inputenc}
\usepackage[font=footnotesize]{caption}
\usepackage{setspace}
\usepackage{threeparttable}
\usepackage{fancyhdr}
\usepackage{floatrow}
\usepackage[outercaption]{sidecap}    
\usepackage{ghsystem}
\renewcommand{\thefootnote}{\roman{footnote}}
\usepackage{url} 
\usepackage[T1]{fontenc}
\usepackage{textcomp}
\usepackage{mathptmx}
\usepackage{multicol}
\usepackage[labelfont=bf]{caption}
\usepackage{listings}
\usepackage{subcaption}
\usepackage[]{mcode}
\lstset{basicstyle=\ttfamily}
\usepackage{ghsystem}
\usepackage[]{mcode}




%Selbst definierte Kommandos
\newcommand{\C}{~$^\circ\text{C}$}
\newcommand{\wn}{$\mathrm{cm^{-1}\ }$}

\newcommand{\graph}[4] 
	{\begin{figure}[H]
		\centering
		\includegraphics[width={#2} \textwidth]{#1} 
		\caption{#3}
		\label{#4}
		\end{figure}}

\newcommand{\J}[2]{$^{#1}J_{\text{#2}}$}
\newcommand{\pr}{~}


\setlength\parindent{0pt}
\setcounter{secnumdepth}{0}
%HAUPTDOKUMENT
\begin{document}
\selectlanguage{german}



%TITELBLATT
\begin{titlepage}

	\thispagestyle{empty}
	
	
	 ETH Z�rich \\
	 Fr�hlingssemester 2016 \\[2cm]

	\begin{center}
	{\LARGE  Praktikum Physikalische und Analytische Chemie   }\\
	\end{center}

	\begin{center}
	{\large Department Chemie und Angewandte Biowissenschaften    }\\
	 Assistent: Romain Dubey \\
	\end{center}
	
	\vspace{20pt}
	
		\begin{center}
		\begingroup
	{\Large \textbf{Blitzlichtphotolytische Untersuchung der L�sungsmittelabh�ngigkeit der Kinetik eines Chromophors und der Bestimmung der Aktivierungsenergie nach Arrhenius  }} \\
	\endgroup
	\end{center}


\vspace{10pt}

	

	
	 \textbf{Abstract}: Mittels Methoden der Blitzlichtphotolyse wurde das photochrome Molek\"ul THBI\footnote{1',3',3'-Trimethyl-6-hydroxyspiro[2H-1-benzopyran-2,2'-indolin} hinsichtlich seiner kinetischen Eigenschaften untersucht. Durch Berechnung der Geschwindigkeitskonstanten bei verschiedenen Temperaturen konnte eine Aktivierungsenergie von $60 \pm 3$ $kJ\cdot mol^{{-1}}$ f\"ur das System  berechnet werden. Messung der Geschwindingkeitskonstanten der Relaxationsreaktion in verschiedenen Ethanol-/Toluol-Mischungen haben ergeben, dass mit zunehmender Polarit\"at des L\"osungsmittelgemsiches die Reaktion der Geschwindigkeit abnimmt. Die Abh\"angigkeit der Geschwindigkeitskonstante vom Toluolgehalt l\"asst sich durch die Funktion $k(x_{Toluene}) =   0.71 + 0.04\cdot \exp (5.67\cdot x_{Toluene})$ beschreiben.
	
\vspace{80pt}
	
	
Z�rich, \today  \\[1cm]
Dominik Z�rcher \hfill Tobias Seidler \\
\texttt{dominizu@student.ethz.ch} \hfill \texttt{tseidler@student.ethz.ch}

	

	
	
	
\end{titlepage}
hallo world

\end{document}